\hypertarget{stm32f4xx__hal__uart_8c}{}\section{Drivers/\+S\+T\+M32\+F4xx\+\_\+\+H\+A\+L\+\_\+\+Driver/\+Src/stm32f4xx\+\_\+hal\+\_\+uart.c File Reference}
\label{stm32f4xx__hal__uart_8c}\index{Drivers/\+S\+T\+M32\+F4xx\+\_\+\+H\+A\+L\+\_\+\+Driver/\+Src/stm32f4xx\+\_\+hal\+\_\+uart.\+c@{Drivers/\+S\+T\+M32\+F4xx\+\_\+\+H\+A\+L\+\_\+\+Driver/\+Src/stm32f4xx\+\_\+hal\+\_\+uart.\+c}}


U\+A\+RT H\+AL module driver. This file provides firmware functions to manage the following functionalities of the Universal Asynchronous Receiver Transmitter (U\+A\+RT) peripheral\+:  


{\ttfamily \#include \char`\"{}stm32f4xx\+\_\+hal.\+h\char`\"{}}\newline


\subsection{Detailed Description}
U\+A\+RT H\+AL module driver. This file provides firmware functions to manage the following functionalities of the Universal Asynchronous Receiver Transmitter (U\+A\+RT) peripheral\+: 

\begin{DoxyAuthor}{Author}
M\+CD Application Team
\begin{DoxyItemize}
\item Initialization and de-\/initialization functions
\item IO operation functions
\item Peripheral Control functions ~\newline
 + Peripheral State and Errors functions ~\newline
 \begin{DoxyVerb}==============================================================================
                      ##### How to use this driver #####
==============================================================================
[..]
  The UART HAL driver can be used as follows:
  
  (#) Declare a UART_HandleTypeDef handle structure.

  (#) Initialize the UART low level resources by implementing the HAL_UART_MspInit() API:
      (##) Enable the USARTx interface clock.
      (##) UART pins configuration:
          (+++) Enable the clock for the UART GPIOs.
          (+++) Configure these UART pins as alternate function pull-up.
      (##) NVIC configuration if you need to use interrupt process (HAL_UART_Transmit_IT()
           and HAL_UART_Receive_IT() APIs):
          (+++) Configure the USARTx interrupt priority.
          (+++) Enable the NVIC USART IRQ handle.
      (##) DMA Configuration if you need to use DMA process (HAL_UART_Transmit_DMA()
           and HAL_UART_Receive_DMA() APIs):
          (+++) Declare a DMA handle structure for the Tx/Rx stream.
          (+++) Enable the DMAx interface clock.
          (+++) Configure the declared DMA handle structure with the required 
                Tx/Rx parameters.                
          (+++) Configure the DMA Tx/Rx Stream.
          (+++) Associate the initialized DMA handle to the UART DMA Tx/Rx handle.
          (+++) Configure the priority and enable the NVIC for the transfer complete 
                interrupt on the DMA Tx/Rx Stream.

  (#) Program the Baud Rate, Word Length, Stop Bit, Parity, Hardware 
      flow control and Mode(Receiver/Transmitter) in the Init structure.

  (#) For the UART asynchronous mode, initialize the UART registers by calling
      the HAL_UART_Init() API.
  
  (#) For the UART Half duplex mode, initialize the UART registers by calling 
      the HAL_HalfDuplex_Init() API.
  
  (#) For the LIN mode, initialize the UART registers by calling the HAL_LIN_Init() API.
  
  (#) For the Multi-Processor mode, initialize the UART registers by calling 
      the HAL_MultiProcessor_Init() API.
      
   [..] 
     (@) The specific UART interrupts (Transmission complete interrupt, 
          RXNE interrupt and Error Interrupts) will be managed using the macros
          __HAL_UART_ENABLE_IT() and __HAL_UART_DISABLE_IT() inside the transmit 
          and receive process.
        
   [..] 
     (@) These APIs (HAL_UART_Init() and HAL_HalfDuplex_Init()) configure also the 
          low level Hardware GPIO, CLOCK, CORTEX...etc) by calling the customized 
          HAL_UART_MspInit() API.
        
   [..] 
      Three operation modes are available within this driver :     

   *** Polling mode IO operation ***
   =================================
   [..]    
     (+) Send an amount of data in blocking mode using HAL_UART_Transmit() 
     (+) Receive an amount of data in blocking mode using HAL_UART_Receive()
     
   *** Interrupt mode IO operation ***    
   ===================================
   [..]    
     (+) Send an amount of data in non blocking mode using HAL_UART_Transmit_IT() 
     (+) At transmission end of transfer HAL_UART_TxCpltCallback is executed and user can 
          add his own code by customization of function pointer HAL_UART_TxCpltCallback
     (+) Receive an amount of data in non blocking mode using HAL_UART_Receive_IT() 
     (+) At reception end of transfer HAL_UART_RxCpltCallback is executed and user can 
          add his own code by customization of function pointer HAL_UART_RxCpltCallback
     (+) In case of transfer Error, HAL_UART_ErrorCallback() function is executed and user can 
          add his own code by customization of function pointer HAL_UART_ErrorCallback

   *** DMA mode IO operation ***    
   ==============================
   [..] 
     (+) Send an amount of data in non blocking mode (DMA) using HAL_UART_Transmit_DMA() 
     (+) At transmission end of half transfer HAL_UART_TxHalfCpltCallback is executed and user can 
          add his own code by customization of function pointer HAL_UART_TxHalfCpltCallback 
     (+) At transmission end of transfer HAL_UART_TxCpltCallback is executed and user can 
          add his own code by customization of function pointer HAL_UART_TxCpltCallback
     (+) Receive an amount of data in non blocking mode (DMA) using HAL_UART_Receive_DMA() 
     (+) At reception end of half transfer HAL_UART_RxHalfCpltCallback is executed and user can 
          add his own code by customization of function pointer HAL_UART_RxHalfCpltCallback 
     (+) At reception end of transfer HAL_UART_RxCpltCallback is executed and user can 
          add his own code by customization of function pointer HAL_UART_RxCpltCallback
     (+) In case of transfer Error, HAL_UART_ErrorCallback() function is executed and user can 
          add his own code by customization of function pointer HAL_UART_ErrorCallback
     (+) Pause the DMA Transfer using HAL_UART_DMAPause()      
     (+) Resume the DMA Transfer using HAL_UART_DMAResume()  
     (+) Stop the DMA Transfer using HAL_UART_DMAStop()      
  
   *** UART HAL driver macros list ***
   ============================================= 
   [..]
     Below the list of most used macros in UART HAL driver.
     
    (+) __HAL_UART_ENABLE: Enable the UART peripheral 
    (+) __HAL_UART_DISABLE: Disable the UART peripheral     
    (+) __HAL_UART_GET_FLAG : Check whether the specified UART flag is set or not
    (+) __HAL_UART_CLEAR_FLAG : Clear the specified UART pending flag
    (+) __HAL_UART_ENABLE_IT: Enable the specified UART interrupt
    (+) __HAL_UART_DISABLE_IT: Disable the specified UART interrupt
    (+) __HAL_UART_GET_IT_SOURCE: Check whether the specified UART interrupt has occurred or not
    
   [..] 
     (@) You can refer to the UART HAL driver header file for more useful macros \end{DoxyVerb}

\end{DoxyItemize}
\end{DoxyAuthor}
\begin{DoxyAttention}{Attention}

\end{DoxyAttention}
\subsubsection*{\begin{center}\copyright{} C\+O\+P\+Y\+R\+I\+G\+H\+T(c) 2017 S\+T\+Microelectronics\end{center} }

Redistribution and use in source and binary forms, with or without modification, are permitted provided that the following conditions are met\+:
\begin{DoxyEnumerate}
\item Redistributions of source code must retain the above copyright notice, this list of conditions and the following disclaimer.
\item Redistributions in binary form must reproduce the above copyright notice, this list of conditions and the following disclaimer in the documentation and/or other materials provided with the distribution.
\item Neither the name of S\+T\+Microelectronics nor the names of its contributors may be used to endorse or promote products derived from this software without specific prior written permission.
\end{DoxyEnumerate}

T\+H\+IS S\+O\+F\+T\+W\+A\+RE IS P\+R\+O\+V\+I\+D\+ED BY T\+HE C\+O\+P\+Y\+R\+I\+G\+HT H\+O\+L\+D\+E\+RS A\+ND C\+O\+N\+T\+R\+I\+B\+U\+T\+O\+RS \char`\"{}\+A\+S I\+S\char`\"{} A\+ND A\+NY E\+X\+P\+R\+E\+SS OR I\+M\+P\+L\+I\+ED W\+A\+R\+R\+A\+N\+T\+I\+ES, I\+N\+C\+L\+U\+D\+I\+NG, B\+UT N\+OT L\+I\+M\+I\+T\+ED TO, T\+HE I\+M\+P\+L\+I\+ED W\+A\+R\+R\+A\+N\+T\+I\+ES OF M\+E\+R\+C\+H\+A\+N\+T\+A\+B\+I\+L\+I\+TY A\+ND F\+I\+T\+N\+E\+SS F\+OR A P\+A\+R\+T\+I\+C\+U\+L\+AR P\+U\+R\+P\+O\+SE A\+RE D\+I\+S\+C\+L\+A\+I\+M\+ED. IN NO E\+V\+E\+NT S\+H\+A\+LL T\+HE C\+O\+P\+Y\+R\+I\+G\+HT H\+O\+L\+D\+ER OR C\+O\+N\+T\+R\+I\+B\+U\+T\+O\+RS BE L\+I\+A\+B\+LE F\+OR A\+NY D\+I\+R\+E\+CT, I\+N\+D\+I\+R\+E\+CT, I\+N\+C\+I\+D\+E\+N\+T\+AL, S\+P\+E\+C\+I\+AL, E\+X\+E\+M\+P\+L\+A\+RY, OR C\+O\+N\+S\+E\+Q\+U\+E\+N\+T\+I\+AL D\+A\+M\+A\+G\+ES (I\+N\+C\+L\+U\+D\+I\+NG, B\+UT N\+OT L\+I\+M\+I\+T\+ED TO, P\+R\+O\+C\+U\+R\+E\+M\+E\+NT OF S\+U\+B\+S\+T\+I\+T\+U\+TE G\+O\+O\+DS OR S\+E\+R\+V\+I\+C\+ES; L\+O\+SS OF U\+SE, D\+A\+TA, OR P\+R\+O\+F\+I\+TS; OR B\+U\+S\+I\+N\+E\+SS I\+N\+T\+E\+R\+R\+U\+P\+T\+I\+ON) H\+O\+W\+E\+V\+ER C\+A\+U\+S\+ED A\+ND ON A\+NY T\+H\+E\+O\+RY OF L\+I\+A\+B\+I\+L\+I\+TY, W\+H\+E\+T\+H\+ER IN C\+O\+N\+T\+R\+A\+CT, S\+T\+R\+I\+CT L\+I\+A\+B\+I\+L\+I\+TY, OR T\+O\+RT (I\+N\+C\+L\+U\+D\+I\+NG N\+E\+G\+L\+I\+G\+E\+N\+CE OR O\+T\+H\+E\+R\+W\+I\+SE) A\+R\+I\+S\+I\+NG IN A\+NY W\+AY O\+UT OF T\+HE U\+SE OF T\+H\+IS S\+O\+F\+T\+W\+A\+RE, E\+V\+EN IF A\+D\+V\+I\+S\+ED OF T\+HE P\+O\+S\+S\+I\+B\+I\+L\+I\+TY OF S\+U\+CH D\+A\+M\+A\+GE. 