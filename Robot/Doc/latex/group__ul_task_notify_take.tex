\hypertarget{group__ul_task_notify_take}{}\section{ul\+Task\+Notify\+Take}
\label{group__ul_task_notify_take}\index{ul\+Task\+Notify\+Take@{ul\+Task\+Notify\+Take}}
task. h 
\begin{DoxyPre}uint32\_t ulTaskNotifyTake( BaseType\_t xClearCountOnExit, TickType\_t xTicksToWait );\end{DoxyPre}


config\+U\+S\+E\+\_\+\+T\+A\+S\+K\+\_\+\+N\+O\+T\+I\+F\+I\+C\+A\+T\+I\+O\+NS must be undefined or defined as 1 for this function to be available.

When config\+U\+S\+E\+\_\+\+T\+A\+S\+K\+\_\+\+N\+O\+T\+I\+F\+I\+C\+A\+T\+I\+O\+NS is set to one each task has its own private \char`\"{}notification value\char`\"{}, which is a 32-\/bit unsigned integer (uint32\+\_\+t).

Events can be sent to a task using an intermediary object. Examples of such objects are queues, semaphores, mutexes and event groups. Task notifications are a method of sending an event directly to a task without the need for such an intermediary object.

A notification sent to a task can optionally perform an action, such as update, overwrite or increment the task\textquotesingle{}s notification value. In that way task notifications can be used to send data to a task, or be used as light weight and fast binary or counting semaphores.

ul\+Task\+Notify\+Take() is intended for use when a task notification is used as a faster and lighter weight binary or counting semaphore alternative. Actual Free\+R\+T\+OS semaphores are taken using the x\+Semaphore\+Take() A\+PI function, the equivalent action that instead uses a task notification is ul\+Task\+Notify\+Take().

When a task is using its notification value as a binary or counting semaphore other tasks should send notifications to it using the x\+Task\+Notify\+Give() macro, or x\+Task\+Notify() function with the e\+Action parameter set to e\+Increment.

ul\+Task\+Notify\+Take() can either clear the task\textquotesingle{}s notification value to zero on exit, in which case the notification value acts like a binary semaphore, or decrement the task\textquotesingle{}s notification value on exit, in which case the notification value acts like a counting semaphore.

A task can use ul\+Task\+Notify\+Take() to \mbox{[}optionally\mbox{]} block to wait for a the task\textquotesingle{}s notification value to be non-\/zero. The task does not consume any C\+PU time while it is in the Blocked state.

Where as x\+Task\+Notify\+Wait() will return when a notification is pending, ul\+Task\+Notify\+Take() will return when the task\textquotesingle{}s notification value is not zero.

See \href{http://www.FreeRTOS.org/RTOS-task-notifications.html}{\tt http\+://www.\+Free\+R\+T\+O\+S.\+org/\+R\+T\+O\+S-\/task-\/notifications.\+html} for details.


\begin{DoxyParams}{Parameters}
{\em x\+Clear\+Count\+On\+Exit} & if x\+Clear\+Count\+On\+Exit is pd\+F\+A\+L\+SE then the task\textquotesingle{}s notification value is decremented when the function exits. In this way the notification value acts like a counting semaphore. If x\+Clear\+Count\+On\+Exit is not pd\+F\+A\+L\+SE then the task\textquotesingle{}s notification value is cleared to zero when the function exits. In this way the notification value acts like a binary semaphore.\\
\hline
{\em x\+Ticks\+To\+Wait} & The maximum amount of time that the task should wait in the Blocked state for the task\textquotesingle{}s notification value to be greater than zero, should the count not already be greater than zero when ul\+Task\+Notify\+Take() was called. The task will not consume any processing time while it is in the Blocked state. This is specified in kernel ticks, the macro pd\+M\+S\+\_\+\+T\+O\+\_\+\+T\+I\+C\+S\+K( value\+\_\+in\+\_\+ms ) can be used to convert a time specified in milliseconds to a time specified in ticks.\\
\hline
\end{DoxyParams}
\begin{DoxyReturn}{Returns}
The task\textquotesingle{}s notification count before it is either cleared to zero or decremented (see the x\+Clear\+Count\+On\+Exit parameter). 
\end{DoxyReturn}
